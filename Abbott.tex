\documentclass[12pt,twoside]{article}
\usepackage[left=1in, right=1in, top=1in, bottom=1in]{geometry} % General margins
\usepackage{changepage} % Allows margin adjustments for specific sections
\usepackage{amsmath, amsthm, amssymb }
\usepackage{graphicx}
\usepackage{caption}
\usepackage{enumitem}
\usepackage{xcolor} % For color definitions
\usepackage[hidelinks]{hyperref}
\usepackage[pagecolor=white]{pagecolor} % For page background color
\usepackage[most]{tcolorbox}
\usepackage{array}
\usepackage{booktabs} % For better-looking table rules 
\usepackage[table]{xcolor}
\usepackage{adjustbox} % For resizing tables
\usepackage{longtable}
\usepackage[version=4]{mhchem}
\usepackage{float}
\renewcommand{\arraystretch}{1.5}
\setlength{\tabcolsep}{10pt}
\tcbuselibrary{listingsutf8}
\definecolor{customgreen}{HTML}{2E6F40}
\definecolor{lightergreen}{HTML}{9b9b9b}
\definecolor{blue}{HTML}{125eaa}
\definecolor{green}{HTML}{066d41}
\definecolor{red}{HTML}{70000e}


\numberwithin{equation}{section} % Number equations by section
\numberwithin{figure}{section}
\numberwithin{table}{section}
\newtheorem{definition}{Definition}[section] % Definitions numbered by section
\newcommand{\reportauthor}{Zach Mollatt - Notes from Stephen Abbott's Understanding Analysis 2$^{\text{nd}}$ edition}
\newcommand{\reporttitle}{Analysis}
\tcbset{
    colframe=customgreen, % Frame colour
    colback=lightergreen, % Background colour
    coltitle=black,       % Title text colour
    coltext=black         % Body text colour
}
%%%%%%%%%%%%%%%%%%%%%%%%%%%%

\begin{document}

% Last modification: 2016-09-29 (Marc Deisenroth)
\begin{titlepage}

    \newcommand{\HRule}{\rule{\linewidth}{0.5mm}} % Defines a new command for the horizontal lines, change thickness here
    
    \null\vfill
    \begin{center} % Center remainder of the page
    %----------------------------------------------------------------------------------------
    %	NAME
    %----------------------------------------------------------------------------------------
    
    %----------------------------------------------------------------------------------------
    %	TITLE SECTION
    %----------------------------------------------------------------------------------------
    
    \HRule \\[0.4cm]
    { \huge \bfseries \reporttitle}\\ % Title of your document
    \HRule \\[1.5cm]
    \end{center}
    %----------------------------------------------------------------------------------------
    %	AUTHOR SECTION
    %---------------------------------------------------------------------------------------
    \begin{flushleft} \large
    \textit{}\\
    \reportauthor~% Your name
    \end{flushleft}
    
    \vfill\null
    
    
    
    \makeatother
    
    
    \end{titlepage}
    
    

\tableofcontents
\newpage

\begin{adjustwidth}{0in}{0in} 
\section{The Real Numbers}
\subsection{Irrationality}
\textbf{The Proof for $\sqrt{2}$ being Irrational}\\
First, using the defintion for a rational number, assume $\sqrt{2}$ can be expressed in the form $\frac{p}{q}$ 
where p and q are integers.\\
Squaring both sides we then get:
\begin{equation}
    \frac{p^2}{q^2} = 2
    \label{eqn:1}
\end{equation}
Next, we will assume that $p$ and $q$ have no common factor as we could cancel it. As a result this implies 
\begin{equation}
   p^2 = 2q^2 
   \label{eqn:2}
\end{equation}
From this we can deduce that $p^2$ is even due to the being divisible by two, meaning $p$ is also even. As a result
we can represent $p = 2r$. 
\begin{equation}
    \therefore 2r^2 = p^2
    \label{eqn:3}
\end{equation}
However, this implies $p$ and $q$ are both even. This is contradictory to our statement earlier assuming that they had no common factor.
Hence equation \eqref{eqn:1} cannot hold and hence, the theorem of irrationality is proven.\\
\textbf{Number Sets}
\begin{itemize}
    \item $\mathbb{N} = \{1, 2, 3, \ldots\}$ — Natural numbers\\
    Natural Numbers can perform addition perfectly well.
    \item $\mathbb{Z} = \{\ldots, -2, -1, 0, 1, 2, \ldots\}$ — Integers\\
    Extend to integers to have an additive identity (0) and subtraction.
    \item $\mathbb{Q} = \left\{ \frac{p}{q} \mid p, q \in \mathbb{Z},\, q \neq 0 \right\}$ — Rational numbers\\
    Multiplication and Division are now capable with this set.
    \item $\mathbb{R}$ — Real Numbers
    This accounts for any "gaps" on the number line where irrational components may be found.
\end{itemize}
$\mathbb{Q}$ defines a field (any set where,  addition and multiplication are well-defined operations
that are commutative, associative, and obey the distributive property $a(b + c) = ab + ac$).


\end{adjustwidth}  



\end{document}
